\documentclass[pdftex,12pt,a4paper]{article}

\usepackage[pdftex]{graphicx}
\usepackage{paralist}
\usepackage{mathtools}
\usepackage{hyperref}
\usepackage[xindy]{glossaries}
\usepackage[utf8]{inputenc}
\usepackage[ngerman]{babel}
\usepackage{fancyhdr}
\usepackage{lastpage}
\usepackage[backend=biber]{biblatex}
\addbibresource{./bibliography.bib}
\usepackage[ngerman]{babel}
\usepackage[babel,german=quotes]{csquotes}

\pagestyle{fancy}
\fancyhf{}
\fancyhead[L]{
	\rotatebox{0}{\scalebox{0.5}[0.5]{\includegraphics{images/BFH_Logo_C_de_100_RGB.png}}}
}
\fancyhead[C]{}
\fancyhead[R]{}
\renewcommand{\headrulewidth}{0.4pt}
\fancyfoot[R]{\thepage /  \pageref{LastPage}}
\renewcommand{\footrulewidth}{0.4pt}

\newcommand{\HRule}{\rule{\linewidth}{0.5mm}}

\newglossaryentry{FooBar}
{
  name=FooBar,
  description={is the nicest of all glossary entries you may ever see in the whole wide world of glossaries.}
}

\makeglossaries


\begin{document}
%deckblatt start
\thispagestyle{empty}
\begin{center}
\rotatebox{0}{\scalebox{1.0}[1.0]{\includegraphics{images/BFH_Logo_C_de_100_RGB.png}}}\\
\end{center}
\begin{center}
\Large{Berner Fachhochschule (BFH)}\\
\end{center}
 
 
\begin{center}
\Large{Departement Informatik}
\end{center}
\begin{verbatim}
\end{verbatim}
\begin{center}
\textbf{\LARGE{BTI7311 - Informatik Seminar}}
\end{center}
\begin{verbatim}
 
 
\end{verbatim}
\begin{center}
\textbf{Bericht}
\end{center}
\begin{verbatim} 
\end{verbatim}
 
\begin{flushleft}
\begin{tabular}{lll}
\textbf{Thema:} & & Voronoidiagramme und Delaunay-Triangulation\\
& & \\
& & \\
& & \\
\textbf{Student:} & & Sven Osterwalder (ostes2@bfh.ch)\\
& & \\
& & \\
\textbf{Date:} & & {\today}\\
& & \\
& & \\
\textbf{Professor:} & & Prof. Pierre Fierz
\end{tabular}
\end{flushleft}

\newpage

\section{Einleitung}

[Eine Einleitung zu dieser Arbeit. Kurzbeschreibung des Ablaufes / der Einteilung.]\\

\noindent [TODO:]
\begin{compactitem}
	\item 10-20 Seiten
	\item Vollständige Literaturliste
	\item Klarer, verständlicher Aufbau
	\item Korrekte Sprache
\end{compactitem}

\newpage

\section{Grundlagen}

[Beschreibung der Grundlagen. Was sind Voronoi-Diagramme, was ist die Delaunay-Triangulation?]

\newpage

\section{Voronoi-Diagramme}

\subsection{Einführung}
[Eine Erklärung was Voronoi-Diagramme sind und wo sie angewendet werden.]

\subsection{Algorithmen}0
[Vorstellung von versch. Algorithmen für Voronoi-Diagramme (sofern mehrere existieren), Laufzeitverhalten, Komplexität, Vor- und Nachteile.]

\subsection{Verallgemeinerte Form}
[Beschreibung der verallgemeinerten Form von Voronoi-Diagrammen.]

\subsection{Praktische Anwendung}
[Ausblick auf praktische Anwendungen und Implementationen, ggf. eigene Implementation.]

\newpage

\section{Delaunay-Triangulation}

\subsection{Einführung}
[Eine Erklärung was Voronoi-Diagramme sind und wo sie angewendet werden.]

\subsection{Algorithmen}
[Vorstellung von versch. Algorithmen für Voronoi-Diagramme (sofern mehrere existieren), Laufzeitverhalten, Komplexität, Vor- und Nachteile.]

\subsection{Zusammenhang mit Voronoi-Diagrammen}
[Erklärung der Dualität von Voronoi-Diagrammen und der Delaunay-Triangulation.]

\subsection{Praktische Anwendung}
[Ausblick auf praktische Anwendungen und Implementationen, ggf. eigene Implementation.]

\newpage

\section{Schlusswort}
[Zusammenfassendes Schlusswort der Arbeit.]

\newpage

\section{Literaturliste}
\printbibliography

\newpage

\section{Glossar}
\printglossary[numberedsection]

\end{document}