\section{Einleitung}
Die vorliegende Arbeit wurde im Rahmen des Moduls BTI7311 - Informatik-\\
seminar - zum Thema \textit{Voronoidiagramme und Delaunay-Triangulation} verfasst.


Der erste Teil gibt dabei einen Überblick der wichtigsten Defintionen sowie Grundlagen der Themen, 
im zweiten Teil finden Sie Details zu den einzelnen Themen.

Bei \textbf{Voronoi-Diagrammen} handelt es sich um eine Zerlegung des Raumes in Regionen: Aus einem endlichen Satz von eindeutigen, isolierten Punkten in einem kontinuierlichen Raum,
wird jeder Ort dieses Raumes dem nächsten Punkt des Satzes zugeordnet.

Verbindet man in einem \textit{m}-dimensionalen Raum die Punkte, deren Regionen eine (\textit{m}-1)-dimensionale Fläche teilen, so erhält man die so genannte \textbf{Delaunay-Triangulation},
welche dual zu dem zugehörigen Voronoidiagramm ist.

Diese schriftliche Arbeit dient als Ergänzung zum Vortrag in diesem Modul.
Der Text kann jedoch auch eigenständig betrachtet werden.

\noindent [TODO:]
\begin{compactitem}
	\item 10-20 Seiten
	\item Vollständige Literaturliste
	\item Klarer, verständlicher Aufbau
	\item Korrekte Sprache
\end{compactitem}
