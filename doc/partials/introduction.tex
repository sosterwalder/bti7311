\section{Einleitung}
\label{sec:introduction}
Die vorliegende Arbeit wurde im Rahmen des Moduls BTI7311-Informatik-\\
seminar zum Thema \textit{Voronoidiagramme und Delaunay-Triangulation} verfasst.

Der erste Teil gibt dabei einen Überblick der wichtigsten Defintionen sowie Grundlagen der Themen, 
im zweiten Teil finden Sie Details zu den einzelnen Themen.

Nehmen Sie an, Sie befinden sich in einer Stadt in einem fremden Land dessen Sprache Sie nicht sprechen, z.B. in Polen.
Sie müssen jedoch um jeden Preis die nächstgelegene Poststelle finden um ein wichtiges Schreiben zu versenden, dessen
Poststempel der des aktuellen Tages sein muss.

Genau solch ein Problem lässt sich mit einem \textit{\textbf{Voronoi-Diagramm}} lösen --- vorausgesetzt Sie verfügen über eine Karte auf welcher
alle umliegenden Poststellen eingezeichnet sind. In der Realität fragen Sie wahrscheinlich eher eine hilfsbereite Person
nach der nächstgelegenen Poststelle.

Angenommen, Sie verfügen über solch eine Karte und haben die nächste Poststelle bereits gefunden (und somit das Schreiben versendet) und
möchten nun die landschaftliche Situation der Poststellen virtuell grafisch darstellen, weil Sie z.B. in der Geoinformatik tätig sind. Genau hierfür
eignet sich die \textit{\textbf{Delaunay-Triangualation.}}

Diese schriftliche Arbeit dient als Ergänzung zum Vortrag in diesem Modul.
Der Text kann jedoch auch eigenständig betrachtet werden.

\noindent [TODO:]
\begin{compactitem}
	\item 10--20 Seiten
	\item Vollständige Literaturliste
	\item Klarer, verständlicher Aufbau
	\item Korrekte Sprache
\end{compactitem}
