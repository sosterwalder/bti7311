\subsection{Algorithmen}
\label{sub:delaunayAlgorithms}
Wie in~\ref{sub:voronoiAlgorithms} bereits erwähnt, sind die beiden Strukturen Voronoi-Diagramm $V(S)$ und Delaunay-Triangluation $DT(S)$ dual. Dies heisst, dass sich eine Struktur in der Zeit $O(n)$ aus der anderen Struktur ableiten lässt.

Es wird angenommen, dass das Voronoi-Diagramm $V(S)$ gegeben ist. Man wählt nun eine beliebige Voronoi-Kante zweier Regionen des Diagramms, verbindet die Punkte, welche diese Kante erzeugen (Voronoi-Knoten) und erhält so eine Delaunay-Kante. Diese ist senkrecht zu der Voronoi-Kante. Man wiederholt dies für alle Voronoi-Kanten im Voronoi-Diagramm und erhält so eine zweite Tesselation der konvexen Hülle der Voronoi-Knoten.
Ist ein Delaunay-Diagramm ``degeneriert'', ist es u.U. nicht zusammenhängend oder enthält Voronoi-Knoten vom Grad grösser als drei, so handelt es sich nach den oben genannten Schritten nicht um eine Delaunay-Triangluation, sondern um eine Delaunay-Pretriangulation. Diese enthält Polygone bestehend aus vier oder mehr Punkten. Diese Polygone lassen sich nun wiederum in Dreiecke unterteilen, indem man zwei Eckpunkte so verbindet, dass das entstehende Liniensegment kein anderes schneidet. So erhält man wiederum eine Delaunay-Triangluation~\parencite[S. 52 bis 56]{atsuyuki2000spatialtessellations}.

Es gibt diverse Algorithmen um direkt aus einer Punktemenge eine Delaunay-Triangluation zu generieren, es sind dies z.B.

\begin{compactitem}
\item Flip-Verfahren (siehe~\cite[S. 233 und 444]{atsuyuki2000spatialtessellations})
\item Das Verfahren der inkrementellen Konstruktion (siehe~\cite[S. 272]{klein2005algorithmischegeometrie})
\item Divide-and-Conquer-Verfahren (siehe~\ref{ssub:voronoiAlgorithmsDivAndConq})
\item Sweep (siehe~\ref{ssub:voronoiAlgorithmsSweep})
\end{compactitem}

Diese einzeln vorzustellen und auf deren Laufzeitverhalten und Komplexität zu untersuchen würde den Rahmen dieses Berichtes jedoch sprengen, daher wird darauf verzichtet.
