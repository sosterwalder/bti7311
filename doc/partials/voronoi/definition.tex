\subsection{Definition und Struktur}

Der Einfachheit halber wird in diesem Abschnitt der Raum $\mathbb{R}^2$ anstatt der unter ~\ref{subsec:voronoi-introduction} eingeführte Raum $\mathbb{R}^m$ verwendet.

Die Voronoi-Region $VR(p, O)$ besteht aus allen Punkten der Ebene, denen der Punkt $p = {p_x, p_y}$ näher ist als jeder andere Punkt aus der Punktemenge $O$.

Entfernt man nun sämtliche Voronoi-Regionen aus der Ebene, bleiben genau die Punkte des Raumes $\mathbb{R}^2$ übrig, die keinen eindeutigen sondern zwei oder mehr nächste Nachbaren in der Objekt-Menge $O$ besitzen.

Diese Punktemenge ist das \textit{Voronoi-Diagramm} $V(O)$ von $O$. \parencite{klein2005algorithmischegeometrie}
