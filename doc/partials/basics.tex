\section*{Teil 1: Grundlagen}

Bei \textbf{Voronoi-Diagrammen} handelt es sich um eine Zerlegung des Raumes in Regionen: Aus einem endlichen Satz von eindeutigen, isolierten Punkten in einem kontinuierlichen Raum,
wird jeder Ort dieses Raumes dem nächsten Punkt des Satzes zugeordnet.

Verbindet man in einem \textit{m}-dimensionalen Raum die Punkte, deren Regionen eine (\textit{m}-1)-dimensionale Fläche teilen, so erhält man die so genannte \textbf{Delaunay-Triangulation},
welche dual zu dem zugehörigen Voronoi-Diagramm ist.
